\documentclass[12pt, a4paper, twoside]{report}
% --- Pakiety i kodowanie ---
\usepackage[utf8]{inputenc}
\usepackage[T1]{fontenc}
\usepackage[polish]{babel}
\usepackage{mathptmx} 
\usepackage{graphicx} 
\usepackage{titlesec} 
\usepackage{indentfirst}
\usepackage{fancyhdr} % Dla stopki na stronie tytułowej

% --- USTAWIENIA MARGINESÓW ---
\usepackage{geometry}
\geometry{
    a4paper,
    twoside,
    top=2.5cm,
    bottom=2.5cm,
    inner=2.5cm,
    outer=1.5cm,
    bindingoffset=0cm, % TODO: W wersji do druku ustawić na 1cm
    headheight=1.25cm, % Wysokość nagłówka
    footskip=1.35cm,   % Odległość stopki
    includehead=false,
    includefoot=false
}
\usepackage{emptypage}

% --- FORMATOWANIE ROZDZIAŁÓW ---
\titleformat{\chapter}[display]
    {\normalfont\huge\bfseries}
    {\chaptertitlename\ \thechapter}
    {10pt} % Odstęp między "Rozdział X" a tytułem
    {\Huge}
\titlespacing*{\chapter}
    {0pt}     % lewy margines
    {0pt}     % odstęp PRZED tytułem (było domyślnie 50pt)
    {20pt}    % odstęp PO tytule

\begin{document}

% Wyłącz numerację dla stron wstępnych
\pagenumbering{gobble}

% --- STRONA TYTUŁOWA ---
\begin{titlepage}
    \centering
    % Logo
    \includegraphics{logo_polsl.png} 
    
    \vfill % Elastyczny odstęp
    
    % Typ pracy
    {\bfseries \LARGE PROJEKT INŻYNIERSKI \par}
    
    \vspace{1cm} % Stały odstęp między typem a tytułem
    
    % Temat pracy
    {\bfseries \Large „System sterowania adresowalnym oświetleniem LED z wykorzystaniem mikrokontrolera ESP32 oraz aplikacji mobilnej" \par}
    
    \vfill
    
    % Dane
    {\bfseries \Large Dominik Paweł PORĘBSKI \par}
    {\bfseries Nr albumu: 305922 \par}
    
    \vfill
    
    % Kierunek i ścieżka
    \setlength{\parindent}{0pt}
    {\bfseries Kierunek: INFORMATYKA W SYSTEMACH I UKŁADACH ELEKTRONICZNYCH \par}
    {\bfseries Ścieżka dyplomowania: Ścieżka B \par}
    
    \vfill
    \centering
    
    % Promotor
    {\bfseries PROWADZĄCY PRACĘ / PROMOTOR \par}
    dr inż. Michał JELEŃ \par
    
    \vfill
    
    % Jednostka
    {\bfseries KATEDRA ENERGOELEKTRONIKI NAPĘDU ELEKTRYCZNEGO I ROBOTYKI \par}
    {\bfseries WYDZIAŁ ELEKTRYCZNY \par}
    
    % Stopka z miastem i rokiem
    \vfill
    \thispagestyle{fancy}
    \fancyhf{}
    \renewcommand{\headrulewidth}{0pt}
    \fancyfoot[C]{GLIWICE, 2026} % Stopka wyśrodkowana
\end{titlepage}
\cleardoublepage

% --- STRESZCZENIE ---
\thispagestyle{empty} 
\noindent \textbf{Tytuł pracy:} \\
System sterowania adresowalnym oświetleniem LED z wykorzystaniem mikrokontrolera ESP32 oraz aplikacji mobilnej.

\vspace{0.5cm}

\noindent \textbf{Streszczenie:} \\
Wpisać tekst streszczenia...

\vspace{0.5cm}

\noindent \textbf{Słowa kluczowe:} \\
Wpisać słowa kluczowe...

\vspace{1.5cm}
\hrule 
\vspace{1.5cm}

% Część angielska
\noindent \textbf{Thesis title:} \\
Addressable LED Lighting Control System Using ESP32 Microcontroller and Mobile Application.

\vspace{0.5cm}

\noindent \textbf{Abstract:} \\
Provide abstract...

\vspace{0.5cm}

\noindent \textbf{Keywords:} \\
Provide keywords...

\cleardoublepage

% --- SPIS TREŚCI ---
\tableofcontents
\cleardoublepage

% --- TREŚĆ ---
% Numeracja arabska od 1
\pagenumbering{arabic}
\pagestyle{plain} 

\chapter{}


\end{document}